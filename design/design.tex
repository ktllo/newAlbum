\documentclass[a4paper,12pt]{report}
\usepackage[utf8]{inputenc}
\usepackage[margin=0.725in]{geometry}
\usepackage{amsmath}
\usepackage{amssymb}
\usepackage{fancyhdr}
\usepackage{amsfonts}
\usepackage{color}
\usepackage[british]{babel}
\usepackage[colorlinks]{hyperref}
\title{API Design for album system Version 0.1.0}
\author{Lo Kam Tao Leo\\leolo@leolo.org}
\date{\today}
\setcounter{tocdepth}{1}
\setlength{\headheight}{15.2pt}
\pagestyle{fancy}

\newcommand{\see}[1]{See page \pageref{#1} for more information.}
\begin{document}
	\maketitle
	\tableofcontents
	\chapter{Preface}
	The key words ``MUST'', ``MUST NOT'', ``REQUIRED'', ``SHALL'', ``SHALL
	NOT'', ``SHOULD'', ``SHOULD NOT'', ``RECOMMENDED'',  ``MAY'', and
	``OPTIONAL'' in this document are to be interpreted as described in
	RFC 2119.
	
	\section{Special JSON data type}
	\begin{itemize}
		\item \label{type:URI}URI which is a string, but MUST contain an URI, 
	\end{itemize}
	\section{Basic API parameters}
	All the parameter listed here is accepted by all API end point
	
	\begin{tabular}{|l|ll|p{10cm}|}
		\hline
		Name & Type & & Description\\\hline
		sessionID & string &  & The assigned sessionID for this session\\\hline
	\end{tabular}
	\chapter{User Control}
	\section{login.php}
	This API end point allow user log in into the system
	\subsection{Parameters}
	\begin{tabular}{|l|ll|p{10cm}|}
		\hline
		Name & Type & & Description\\\hline
		userid & string & required & User ID for the user\\\hline
		password & string & required & plain-text password of the user\\\hline
		cookie & boolean & optional & 	Is the cookie set upon successful login.\newline
										Default with ``true''\\\hline
	\end{tabular}
	\subsection{Return Value}
	This API call always return HTTP response code 200. With a body in JSON format.
	\subsection{JSON structure}
	\begin{tabular}{|l|l|l|p{10cm}|}
		\hline
		Name & Type & Sub-type & Description\\\hline
		result & string & & 	`SUCCESS' if the login is successful, or\newline
									`FAIL' otherwise\\\hline
		sessionID & string & & The assigned sessionID for this session if login is success\\\hline
		message & string & & Message to be shown to the user. If this field is empty, it will not included in the response. This message is unformatted.\\\hline
	\end{tabular}
	\section{logout.php}
	This API end point allow user log out from the system
	\subsection{Parameters}
	\begin{tabular}{|l|ll|p{10cm}|}
		\hline
		Name & Type & & Description\\\hline
		sessionID & string & required & The assigned sessionID for this session\\\hline
	\end{tabular}
	\subsection{Return Value}
	This API call always return HTTP response code 200. With a body in JSON format.
	\subsection{JSON structure}
	\begin{tabular}{|l|l|l|p{10cm}|}
		\hline
		Name & Type & Sub-type & Description\\\hline
		result & string &  & 	`SUCCESS' if the logout is successful, or\newline
		`FAIL' otherwise\\\hline
	\end{tabular}
	\section{pwdchange.php}
	This API end point allow user changes the password
	\subsection{Parameters}
	\begin{tabular}{|l|ll|p{10cm}|}
		\hline
		Name & Type & & Description\\\hline
		sessionID & string & required & The assigned sessionID for this session\\\hline
		old\_password & string & required & plain-text version of the current password of the user\\\hline
		new\_password & string & required & plain-text version of the new password of the user\\\hline
	\end{tabular}
	\subsection{Return Value}
	This API call always return HTTP response code 200. With a body in JSON format.
	\subsection{JSON structure}
	\begin{tabular}{|l|l|l|p{10cm}|}
		\hline
		Name & Type & Sub-type & Description\\\hline
		result & string & always & 	`SUCCESS' if the password change is successful, or\newline
		`FAIL' otherwise\\\hline
		message & string & & Message to be shown to the user. If this field is empty, it will not included in the response. This message is unformatted.\\\hline
	\end{tabular}
	\chapter{Album Information}
	
	\section{albumlist.php}
	This API end point allow user retrieves a list of album
	\subsection{Parameters}
	\begin{tabular}{|l|ll|p{10cm}|}
		\hline
		Name & Type & & Description\\\hline
		lastID & string & optional & Last album ID retrieved from previous request\\\hline
		maxSize & number & optional & Maximum number of album to be retrieved.\newline
		\textbf{IMPORTANT}: Maximum number may be overridden by the server setting \\\hline
	\end{tabular}
	\subsection{Return Value}
	This API call always return HTTP response code 200. With a body in JSON format.
	\subsection{JSON structure}
	\begin{tabular}{|l|l|l|p{10cm}|}
		\hline
		Name & Type & Sub-type & Description\\\hline
		size & number & & Number of the album included in this response\\\hline
		more & boolean & & Is there any further album available for retrieval\\\hline
		albums & array & album & The albums retrieved from the request. \see{obj:album}\\\hline
	\end{tabular}
	
	\section{listimage.php}
	This API end point allow user retrieves a list of image from an album
	\subsection{Parameters}
	\begin{tabular}{|l|ll|p{10cm}|}
		\hline
		Name & Type & & Description\\\hline
		album & string & required & The target album going to retrieves inages\\\hline
		lastID & string & optional & Last album ID retrieved from previous request\\\hline
		maxSize & number & optional & Maximum number of album to be retrieved.\newline
		\textbf{IMPORTANT}: Maximum number may be overridden by the server setting \\\hline
	\end{tabular}
	\subsection{Return Value}
	This API call always return HTTP response code 200. With a body in JSON format.
	\subsection{JSON structure}
	\begin{tabular}{|l|l|l|p{10cm}|}
		\hline
		Name & Type & Sub-type & Description\\\hline
		size & number & & Number of the album included in this response\\\hline
		more & boolean & & Is there any further album available for retrieval\\\hline
		albums & array & album & The albums retrieved from the request. \see{obj:album}\\\hline
	\end{tabular}
	
	
	
	\chapter{JSON object definition}
	\section{album}\label{obj:album}
	This object carries information about an album
	
	\begin{tabular}{|l|l|l|p{10cm}|}
		\hline
		Name & Type & Sub-type & Description\\\hline
		id & string & & Identifier of the album\\\hline
		name & string & & Name of the album\\\hline
		thumbnail & object & thumbnail & Album thumbnail. \see{obj:thumbnail}\\\hline
		desc & string & & Description for this album\\\hline
		owner & string & & Owner's name for this album\\\hline
		size& number & & Number of images in the album\\\hline
	\end{tabular}
	\section{thumbnail}\label{obj:thumbnail}
	This object carries information about an thumbnail
	
	\begin{tabular}{|l|l|l|p{10cm}|}
		\hline
		Name & Type & Sub-type & Description\\\hline
		id & string & & Identifier of the thumbnail\\\hline
		photoId & string & & Which photo this thumbnail is for\\\hline
		width & number & & Width of the thumbnail in pixel\\\hline
		height & number & & Height of the thumbnail in pixel\\\hline
		url & URI & & URL for the thumbnail, MUST be relative to the application base\\\hline
	\end{tabular}
	\section{photo}\label{obj:photo}
	This object carries information about an photo
	
	\begin{tabular}{|l|l|l|p{10cm}|}
		\hline
		Name & Type & Sub-type & Description\\\hline
		id & string & & Identifier of the photo\\\hline
		name & string & & File name of the photo\\\hline
		title & string & & Title of the photo\\\hline
		desc & string & & Description of the photo \\\hline
		width & number & & Width of the photo in pixel\\\hline
		height & number & & Height of the photo in pixel\\\hline
		thumbnails & object & thumbnail & Selected thumbnail of the photo, \see{obj:thumbnail}\\\hline
		size & number & & Size in bytes for the full photo\\\hline
		url & URI & & URL for the photo, MUST be relative to the application base\\\hline
	\end{tabular}
\end{document}
